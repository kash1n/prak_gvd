\documentclass[specialist, subf, href, colorlinks=true, 14pt, final]{disser}
\usepackage[a4paper, mag=1000, includefoot, left=3cm, right=1.5cm, top=2cm, bottom=2cm, headsep=1cm, footskip=1cm]{geometry}

\usepackage[T2A]{fontenc}
\usepackage [utf8] {inputenc}
\usepackage[english, russian]{babel}
\usepackage{caption}
\usepackage{enumerate}
\usepackage{amsmath,amsthm,amssymb}
\usepackage{wrapfig}
\usepackage{makecell}
%\usepackage {enumitem}  
%\usepackage{graphicx}
%\usepackage{multicol}
\usepackage{mathrsfs}
\usepackage{xcolor}
\usepackage{tikz}
\usetikzlibrary{decorations.pathreplacing}
\setcounter{tocdepth}{2}
%\usepackage{hyperref}
%\usepackage{algorithm}
%\usepackage[noend]{algpseudocode}
%\usepackage[margin=1in]{geometry}

\theoremstyle{definition}
\newtheorem{defn}{Определение}[section]
\newtheorem{example}{Пример}[section]
\newtheorem{state}{Утверждение}[section]
\newtheorem{theorem}{Теорема}[section]
\newtheorem{lemma}{Лемма}[section]
\newtheorem{axiom}{Аксиома}[section]
\newtheorem{consequence}{Следствие}[section]

\newcommand{\anonsection}[1]{\section*{#1}\addcontentsline{toc}{section}{#1}}
\newcommand{\anonsubsection}[1]{\subsection*{#1}\addcontentsline{toc}{subsection}{#1}}
\newcommand{\pdfrac}[2]{\frac{\partial #1}{\partial #2}}
\newcommand{\const}{\text{const}}
\newcommand{\tang}{\text{tg}}
  
 % Цвета для гиперссылок
\hypersetup{pdfstartview=FitH, linkcolor=blue, urlcolor=blue, colorlinks=true}


\begin{document}

\begin{titlepage}
\begin{center}
МОСКОВСКИЙ ГОСУДАРСТВЕННЫЙ УНИВЕРСИТЕТ\\
имени М. В. Ломоносова\\
Механико-математический факультет\\
\vspace{1cm}
\begin{figure}[!htp]%
  \begin{center}%
        {\includegraphics[width=20mm]{pics/mmlogo.png}}%
  \end{center}
\end{figure}
\vspace{4cm}
\Large{
ЗАДАЧИ ФИЗИКО-МЕХАНИЧЕСКОГО ПРАКТИКУМА ПО ГАЗОВОЙ И ВОЛНОВОЙ ДИНАМИКЕ
}\\
\vspace{4cm}
\end{center}
\end{titlepage}

\tableofcontents

\anonsection{Раздел 1}
\anonsubsection{Задача 1. Продольное соударение упругих стержней}
Целью работы является изучение процесса соударения двух стержней и определение скорости упругой волны и деформации, возникающей в стержнях при ударе.\\
I. \underline{Описание явления}\\
Процесс распространения малых возмущений в изотропной 
упругой среде в общем случае можно изучать, исследуя решения 
уравнений Ляме для конкретных начальных и граничных условий [I]. 
Однако общая задача о распространении волн в ограниченном упругом
пространстве довольно сложна. Сен-Венаном разработана 
приближенная теория продольных волн в длинных тонких стержнях. При этом в
качестве основного предположения принята гипотеза плоских 
сечений, т.е. полагается, что любое плоское сечение, образованное
точками среды и перпендикулярное оси стержня, остается при 
движении плоским и перпендикулярным оси, а напряжение во всех 
точках такого сечения одинаково и меняется только со временем. 
Принятые предположения позволяют рассматривать продольное движение
стержней в одномерной постановке.\\
II. \underline{Теоретическая часть}\\
Рассмотрим соударение двух стержней с равной начальной 
длиной $l_0$ , изготовленных из одного и того же материала и имеющих
одинаковое начальное поперечное сечение $S_0$ (Рис. \ref{1-1-1}).
Пусть недеформированный стержень I движется вдоль своей оси
со скоростью $v_0$ и в момент $t = 0$ касается соосного с ним 
стержня II. Для изучения последующего процесса соударения введем 
неподвижную ось $Ox$ с началом, совпадающим с левым торцом первого
стержня в момент $t = 0$.\\
\begin{figure}[!htp]
  \center{\includegraphics[width=160mm]{pics/1-1-1.png}}
  \caption{}
  \label{1-1-1}
\end{figure}
\\
В качестве лагранжевой координаты рассмотрим начальную 
координату сечения $\xi = x(0)$. Перемещение $u = u(\xi, t)$ можно 
представить в виде
\[
  u(\xi, t) = x(\xi, t) - \xi
\]
Тогда скорость $v$ и продольная деформация $\varepsilon$ в данном сечении $\xi$ соответственно равны
\begin{equation}\label{eq111}
  v = \frac{\partial u}{\partial t},\ \ \varepsilon = \frac{\partial u}{\partial \xi}
\end{equation}
Запишем уравнение движения (второй закон Ньютона) для 
выделенного малого элемента AB (Рис. \ref{1-1-1}), ограниченного сечениями $\xi$ и $\xi + \Delta\xi$:
\begin{equation}\label{eq112}
  \rho\Delta\xi S_{0} \frac{\partial v}{\partial t} = S_{0} \left[\sigma(\xi+\Delta\xi, t) - \sigma(\xi, t)\right],
\end{equation}
где $\rho$ -- начальная плотность материала стержней; $\sigma(\xi,t)$ -- напряжение в данном сечении $\xi$. 
Поделив (\ref{eq112}) на $\Delta\xi$ и сделав предельный переход $(\Delta\xi \rightarrow 0)$, приходим к уравнению
\begin{equation}\label{eq113}
  \rho\frac{\partial v}{\partial t} = \frac{\partial \sigma}{\partial \xi}
\end{equation}
Материал стержней считаем линейно-упругим, т.е.
\begin{equation}\label{eq114}
  \sigma = E\varepsilon,
\end{equation}
где $Е$ - модуль Юнга. 
Уравнения (\ref{eq111}), (\ref{eq113}), (\ref{eq114}) позволяют написать замкнутую систему уравнений для определения скорости и деформации:
\begin{equation}\label{eq115}
  \begin{aligned}
  &\pdfrac{v}{t} = a_{0}^{2} \pdfrac{\varepsilon}{\xi},\\ 
  &\pdfrac{v}{\xi} = \pdfrac{\varepsilon}{t}
  \end{aligned}
\end{equation}
где $a_{0}^{2} = \sqrt{E/\rho}$.
В начальный момент времени $(t = 0)$ первый стержень 
напряжен и имеет скорость $v$, а второй стержень покоится, т.е.
\begin{equation}\label{eq116}
  \begin{aligned}
  &\text{при } t = 0,\ 0 \leqslant \xi < l_{0};\ \varepsilon = 0,\ v = v_{0};\\ 
  &\text{при } t = 0,\ l_{0} < \xi \leqslant 2l_{0};\ \varepsilon = 0,\ v = 0.
  \end{aligned}
\end{equation}
На свободных торцах стержней во всё время движения напряжение равно нулю, поэтому при $t > 0$
\begin{equation}\label{eq117}
  \varepsilon(0, t) = \varepsilon(2l_{0}, t) = 0
\end{equation}
В месте контакта стержней $(\xi = l_{0})$ должны быть равны скорости
и напряжения, т.е. при $t > 0$
\begin{equation}\label{eq118}
  \varepsilon(l_{0}^{-}, t) =\varepsilon(l_{0}^{+}, t),\ \ v(l_{0}^{-}, t) = v(l_{0}^{+}, t)
\end{equation}
Здесь знаки $(-)$, $(+)$ соответствуют значениям функции при подходе слева и справа к точке $\xi = l_0$. Для решения системы (\ref{eq115}) приведем ее к характеристическому виду [\ref{eq112}, \ref{eq113}]. Для этого первое уравнение системы умножим на $dt$, второе -- на $d\xi$ и сложим полученные равенства:
\[
  \pdfrac{v}{t}dt + \pdfrac{v}{\xi}d\xi = a_{0}^{2}\pdfrac{\varepsilon}{\xi}dt + \pdfrac{\varepsilon}{t}d\xi
\]
Левая часть данного уравнения является полным дифференциалом $dv$ для любых $(dt, d\xi)$. Найдем характеристические направления $d\xi = \lambda dt$ такие, вдоль которых и правая часть полученного уравнения является полным дифференциалом. Легко видеть, что таких направлений два $d\xi = \pm a_{0}dt$ в каждой точке фазовой плоскости $(\xi, t)$. \\
Таким образом, мы имеем уравнения характеристик и соотношений на них:
\begin{equation}\label{eq119}
  d\xi = \pm a_{0}dt,\ \ \ dv = \pm a_{0}d\varepsilon
\end{equation}
Полученные соотношения позволяют решить поставленную условиями (\ref{eq116})-(\ref{eq118}) граничную задачу.\\
\begin{figure}[!htp]
  \center{\includegraphics[width=160mm]{pics/1-1-2.png}}
  \caption{}
  \label{1-1-2}
\end{figure}
\\
Покажем это на примере определения решения в точке $M$ (Рис. \ref{1-1-2}). Линии $MN_1$, $MN_2$, $N_{1}N$ являются характеристиками. Сначала определяем решение в граничной точке $N_1$, затем в точке $M$.\\
Из условия (\ref{eq119}) на характеристике $NN_1$ после интегрирования
и использования начальных условий (\ref{eq116}) в точке $N$ получаем $v = -a_{0}\varepsilon + v$. Это уравнение выполняется в любой точке характеристики $NN_1$ и содержит два неизвестных. В точке $N_1$, помимо данного уравнения, должно быть выполнено граничное условие (\ref{eq117}), т.е. в точке $N_1$ мы получили два уравнения для определения скорости и деформации, откуда находим
\begin{equation}\label{eq1110}
  v = v_{0},\ \ \varepsilon = 0.
\end{equation}
Аналогично получаем:
\begin{equation}\label{eq1111}
  \begin{aligned}
  &\text{на характеристике } N_{1}M:\ \ v = a_{0}\varepsilon + v_{0};\\
  &\text{на характеристике } N_{2}M:\ \ v = -a_{0}\varepsilon;
  \end{aligned}
\end{equation}
постоянные интегрирования здесь определены полученным решением в точке $N_1$ (\ref{eq1110}) и начальными условиями (\ref{eq116}) в точке $N_2$ (Рис. \ref{1-1-2}). Из системы (\ref{eq1111}) находим решение в точке $M$: $v = v_{0}/2,\ \varepsilon = -v_{0}/(2a_{0})$. Данный метод решения позволяет определить значения скорости и деформации в любой точке исследуемой области фазовой плоскости. Найденные решения приведены на Рис. \ref{1-1-2}. Область
$OPQG$ разбита характеристиками $AB$, $BC$, $CD$ и $DA$ на пять частей, в каждой из которых решение постоянно. Отметим, что момент времени, соответствующий прямой $PQ$ фазовой плоскости, будет концом соударения,
поскольку в этот момент условия (\ref{eq118}) не выполняются и происходит разлет стержней.\\
Проанализируем полученное решение (Рис. \ref{1-1-2}). В произвольном
сечении $N_{2}K$ второго стержня поведение решения следующее. До 
момента времени, соответствующего точке $K_1$ (момент прихода волны
нагрузки), все параметры сохраняют свои начальные значения. Затем
скачком меняются и остаются постоянными до момента времени, соответствующего точке $K_2$ (момент прихода отраженной от свободного
торца волны разгрузки $CB$). В момент прохождения волны разгрузки параметры решения в данном сечении снова скачком меняются, причем деформация принимает нулевое значение. Данный анализ позволяет сказать, что распространению возмущений в фазовой плоскости соответствуют характеристики. При этом величина $a_0$ является скоростью распространения возмущений но материальным частицам сечений стержня.\\
III. \underline{Экспериментальная часть}\\
Схема экспериментальной установки приведена на Рис. \ref{1-1-3}. 
Плоско-параллельное движение стержней 1 обеспечивается их подвеской на тягах 2. Заданная скорость соударения определяется высотой $h$, с которой сбрасывается стержень
\[
  v_{0} = \sqrt{2gh}
\]
\begin{figure}[!htp]
  \center{\includegraphics[width=160mm]{pics/1-1-3.png}}
  \caption{}
  \label{1-1-3}
\end{figure}
\\
Эксперимент основан на определении поведения деформации в фиксированном сечении второго стержня, в котором наклеивается тензодатчик 3. В момент касания стержней ($t = 0$) замыкается цепь I запуска развертки осциллографа. На вертикальную развертку осциллографа подается сигнал из цепи II тензодатчика 3. Расстояние от середины тензодатчика до свободного торца второго стержня известно и равно $L$.\\
В результате соударения на экране осциллографа наблюдается сигнал (Рис. \ref{1-1-4}), который фиксируется на фотопленку.\\
\begin{wrapfigure}[7]{l}{0.65\linewidth} 
  \includegraphics[width=100mm]{pics/1-1-4.png}
  \caption{}
  \label{1-1-4}
\end{wrapfigure}
\\
На Рис. \ref{1-1-4} точка $A$ соответствует времени подхода волны нагрузки к началу тензодатчика, точка $B$ -- времени его полного нагружения. Точки $C$ и $D$ соответствуют началу разгрузки и полной разгрузке датчика отраженной от свободного торца волной. При этом за время $T$, соответствующее длине средней линии трапеции $ABCD$, волна пробегает дважды расстояние $L$ -- от середины датчика до свободного правого торца второго стержня (рис. \ref{1-1-3}). Поэтому экспериментальное значение скорости распространения возмущений 
определяется выражением
\[
  a = \frac{2L}{T}
\]
Высота трапеции $\Omega$ зависит от изменения напряжения в цепи тензодатчика. Изменение напряжения определяется изменением сопротивления датчика в результате его деформации.\\
При линейном усилении сигнала можно считать связь между скачком $\Omega$ и вызывающей этот скачок деформацией линейной, т.е.
\begin{equation}\label{eq1112}
  \varepsilon = k\Omega
\end{equation}
Тензометрический датчик 3 (рис. \ref{1-1-3}) изготовлен из константовой проволоки диаметром 0.03 мм с измерительной базой в 10 мм. Будучи наклеенным на металлический стержень, датчик воспринимает его деформацию. Деформация и изменение сопротивления датчика связаны соотношением
\begin{equation}\label{eq1113}
  \varepsilon = \frac{1}{S}\frac{\Delta R_{g}}{R_{g}}
\end{equation}
где $S$ -- коэффициент тензочувствительности; $R_g$ -- сопротивление недеформированного датчика; $\Delta R_{g}$ -- изменение сопротивления датчика в результате его деформации. Для определения коэффициента пропорциональности $k$ в формуле (\ref{eq1112}) проводится тарировка. Во время тарировки сопротивление в цепи датчика меняется на известную величину путем включения дополнительного сопротивления $R_T$ (рис. \ref{1-1-5}).
\begin{figure}[!htp]
  \center{\includegraphics[width=160mm]{pics/1-1-5.png}}
  \caption{}
  \label{1-1-5}
\end{figure}
\\
При включении дополнительного тарировочного сопротивления $R_T$ сопротивление меняется на величину
\begin{equation}\label{eq1114}
  \Delta R_{g} = \frac{R_{g}^{2}}{R_{g} + R_{T}},
\end{equation}
что приводит к скачку $\Omega_{T}$ луча осциллографа. Подставляя изменение сопротивления (\ref{eq1114}) в формулу (\ref{eq1113}), определяем соответствующую данному тарировочному изменению сопротивления деформацию
\[
  \varepsilon_{T} = \frac{R_{g}}{S(R_{g}+R_{T})}
\]
и, учитывая, что $\varepsilon_{T} = k\Omega$, находим из (\ref{eq1112}) экспериментальное
значение деформации
\[
  \varepsilon_{\text{Э}} = \varepsilon_{T}\frac{\Omega}{\Omega_{T}}
\]
\\\\\\
IV. \underline{Порядок выполнения работы}\\
1) В начале работы проводится включение и прогрев осциллографа (согласно инструкции по его эксплуатации). В эксперименте используются два луча. На один из них подается сигнал с датчика, второй луч должен быть установлен на высоте тарировочного скачка первого луча. Поэтому в начале работы проводится включение в цепь датчика тарировочного сопротивления и на высоте подскока первого луча устанавливается второй луч.\\
2) Осциллограф переводится в режим работы с ждущей разверткой лучей и проверяется его автоматический запуск в момент касания стержней.\\
3) Стержень 1 отводится назад, тем самым обеспечивается заданная высота подъема.\\
4) Приводится в готовность фотоаппарат. Фотоаппарат устанавливается на произвольную выдержку.\\
5) Работающий на осциллографе нажимает кнопку фотоаппарата и подает команду на сброс стержня. Фиксация заканчивается после удара.\\
6) Опыт повторяется для других высот подъема стержня.\\
\\
V. \underline{Требования к отчёту}\\
Отчет оформляется и виде заполненной таблицы обработки осциллограмм и проведенных экспериментов и сравнения экспериментальных значений деформации и скорости звука с их теоретическими значениями.
\begin{center}
\begin{tabular}{|c|c|c|c|c|c|c|c|c|c|c|c|}
\hline
N &$h$, мм&$v_0$, м/с&$\Omega$, мм&$\Omega_T$, мм& $\varepsilon_T$ & $\varepsilon_{\text{Э}}$ & $\varepsilon$ & $T$, с & $L$, м & $\frac{\varepsilon_{\text{Э}} - \varepsilon}{\varepsilon}$ & $\frac{a - a_{0}}{a_{0}}$\\
\hline
1 &  &  &  &  &  &  &  & &  &  &\\
\hline
2 &  &  &  &  &  &  &  & &  &  &\\
\hline
\end{tabular}
\end{center}
При этом теоретическое значение деформации определяется решением (рис. \ref{1-1-2}).\\
\\
VI. \underline{Упражнения}\\
1) Определить значение деформации в полубесконечном стержне при его ударе о жесткую преграду, если параметры материала стержня $E, \rho$ известны и известна скорость соударения $v_0$.\\
2) Определить коэффициент тензочувствительности $S$ в формуле (\ref{eq1113}) для датчика, представляющего собой цилиндрическую проволоку длиной $l_0$ и радиусом $r_0$ с известными удельным сопротивлением и коэффициентом Пуассона $\nu$ материала проволоки.\\
3) Определить, пользуясь найденным решением (рис. \ref{1-1-2}), закон движения произвольного сечения.\\
\\
VII. \underline{Контрольные вопросы}\\
1) Какие условия на контактной поверхности возникают при соударении стержней из разных материалов?\\
2) Определить погрешность при определении скорости звука $a$, если считать, что погрешность измерения длины равна 1 мм, а погрешность определения времени $10^{-6}$ с.\\
$L = 144.5$ см; $S = 2.9$ -- коэффициент тензочувствительности;\\
$R_{T} = 3000000$ Ом; $R_{g} = 200$ Ом;\\
$t_{\text{разв}} = 200$ мкс/см; $a_{0} = 5100$ м/с -- теоретическое.\\
\begin{center}
ЛИТЕРАТУРА
\end{center}
1. Седов Л.И. Механика сплошной среды. Т.1,2. М.: Наука, 1970.\\
2. Рахматулин Х.А., Демьянов Ю.А. Прочность при интенсивных кратковременных нагрузках. М.: Физматгиз, 1961.\\
3. Гольдсмит В. Удар. М.: Стройиздат, 1965.\\

\anonsubsection{Задача 2. Сверхзвуковое обтекание клина}
%\setcounter{equation}{0}
Целью работы является определение связи между параметрами набегающего потока и параметрами потока, проходящего через косой скачок, возникающий при обтекании тел сверхзвуковым потоком, ознакомление с принципиальной схемой к устройством сверхзвуковой аэродинамической трубы, определение параметров потока в рабочей части трубы по параметрам торможения, а также графическое (методом годографа) решение задачи об обтекании клина сверхзвуковым потоком.\\
I. \underline{Теоретическая часть}\\
1. Если в горизонтальном потоке газа плотностью $\rho$ и скоростью $v$ выделить трубку тока площадью поперечного сечения $\sigma$,
то энергия, переносимая газом массой $m = \rho v \sigma$ через это сечение за секунду складывается из кинетической энергии $mv^{2} / 2$, 
работы сил давления $p \sigma v$ и внутренней энергии $mU$ ($U$ -- внутренняя энергия единицы массы, потенциальная энергия сил тяжести
не учитывается). Полная энергия для единицы массы будет равна
($\rho$ -- плотность):
\addtocounter{equation}{1}
\begin{equation}\label{eq:121}
  E_{e} = \frac{v^2}{2} + U + \frac{p}{\rho}
  \tag{1}
\end{equation}
Уравнение состояния для газа запишется в виде
\addtocounter{equation}{1}
\begin{equation}\label{eq:122}
  p = \rho R T
  \tag{2}
\end{equation}
где $R = C_{p}-C_{v}$; $C_{p}$, $C_{v}$ -- теплоемкости газа при постоянном давлении и постоянном объеме. Для воздуха $R = 288.7\  \text{м}^{2}/\text{с}^{2}\cdot \text{град}$. Величина $U+ p/\rho = C_{p}T = i$ называется теплосодержанием. Учитывая уравнение \eqref{eq:122}, получим
\[
  i = C_{p}\frac{p}{\rho R} = C_{p}\frac{p}{\rho(C_{p}-C{v})} = \frac{p}{\rho}\frac{k}{k - 1}
\]
Здесь $k = C_{p}/C_{v}$ -- показатель адиабаты, для воздуха $k = 1.4$.
Теперь энергию единичной массы газа запишем в виде
\addtocounter{equation}{1}
\begin{equation}\label{eq:121d}
  E_{e} = \frac{v^2}{2} + \frac{k}{k-1}\frac{p}{\rho}
  \tag{1$'$}
\end{equation}
При установившемся течении без вязкости и теплообмена поток энергии вдоль трубки тока сохраняется, поэтому при переходе частицы газа из состояния с параметрами $p_{1},\ \rho_{1},\ v_{1}$ в состояние с 
параметрами $p_{2},\ \rho_{2},\ v_{2}$ выполняется равенство
\addtocounter{equation}{1}
\begin{equation}\label{eq:121dd}
  \frac{v_{1}^{2}}{2}+\frac{k}{k-1}\frac{p_1}{\rho_1} = \frac{v_{2}^{2}}{2}+\frac{k}{k-1}\frac{p_2}{\rho_2}\ \ \text{или}\ \ \frac{v^{2}}{2}+\frac{k}{k-1}\frac{p}{\rho} = \const
  \tag{1$''$}
\end{equation}
Обратим внимание на следующее: если для движущейся частицы газа со скоростью $v$ ее параметры состояния есть давление $p$, плотность $\rho$ и температура $T$, то при адиабатическом (без теплообмена с окружающей средой) торможении до $v = 0$ получим параметры торможения частицы -- давление торможения $p_0$, плотность торможения $\rho_0$ и температуру торможения $T_0$, удовлетворяющие уравнению состояния
\addtocounter{equation}{1}
\begin{equation}\label{eq:122d}
  p_{0} = \rho_{0} R T_{0}
  \tag{2$'$}
\end{equation}
Соответствующие формулы для скорости распространения малых  возмущений или скорости звука будут иметь вид
\addtocounter{equation}{1}
\begin{equation}\label{eq:123}
  a = \sqrt{\frac{dp}{d\rho}} = \sqrt{k\frac{p}{\rho}}\ \ \text{или}\ \ a^{2} = k\frac{p}{\rho} = kRT
  \tag{3}
\end{equation}
\addtocounter{equation}{1}
\begin{equation}\label{eq:123d}
  a_{0} = \sqrt{\frac{dp_{0}}{d\rho_{0}}} = \sqrt{k\frac{p_{0}}{\rho_{0}}}\ \ \text{или}\ \ a_{0}^{2} = k\frac{p_{0}}{\rho_{0}} = kRT_{0}
  \tag{3$'$}
\end{equation}
Для теплосодержания справедливо равенство
\addtocounter{equation}{1}
\begin{equation}\label{eq:124}
  i_{0} = i + v^{2}/2
  \tag{4}
\end{equation}
где $i$ -- теплосодержание частицы единичной массы газа, движущейся со скоростью $v$, a $i_{0} = C_{p}T_{0}$ -- теплосодержание заторможенной частицы (теплосодержание торможения).\\
Из равенства \eqref{eq:124} видно, что при адиабатическом расширении
газа (например, при истечении из сосуда через любой насадок)
скорость расширения (истечения) максимальна при минимальном 
теплосодержании, т.е. когда $i = 0$,
\addtocounter{equation}{1}
\begin{equation}\label{eq:125}
  i_{0} = v_{\text{max}}^{2}/2
  \tag{5}
\end{equation}
В этом случае теплосодержание частицы газа переходит полностью
в ее кинетическую энергию.\\
Далее имеем
\[
  \frac{v^2}{2} = i_{0} - i = C_{p}(T_{0} - T);\ \ \frac{i_{0} - i}{2} = \frac{v^2}{2i}
\]
или
\[
  \frac{T_{0}-T}{T} = \frac{v^2}{2C_{p}T}\cdot\frac{R}{R} = \frac{v^{2}(C_{p}-C-{v})}{2TRC_p} = \frac{k-1}{2}M^{2},
\]
где $M = v/a$ -- число Маха.\\
Итак,
\addtocounter{equation}{1}
\begin{equation}\label{eq:126}
  \frac{T_{0}-T}{T} = \frac{k-1}{2}M^{2}\ \ \text{или}\ \ \frac{T_0}{T} = 1 + \frac{k-1}{2}M^{2}
  \tag{6}
\end{equation}
Если скорость потока (истечения) в некотором месте (сечении канала) равна скорости звука в том же сечении, то такая скорость называется критической. При этом $v_{\text{кр}} = a_{\text{кр}}M = 1$ и формула \eqref{eq:126} примет вид
\addtocounter{equation}{1}
\begin{equation}\label{eq:127}
  T_{0}/T_{\text{кр}} = (k+1)/2\ \ \text{или}\ \ T_{\text{кр}} = 2T_{0}/(k+1)
  \tag{7}
\end{equation}
Отсюда имеем
\addtocounter{equation}{1}
\begin{equation}\label{eq:128}
  a_{0}/a_{\text{кр}} = \sqrt{T_{0}/T_{\text{кр}}} = \sqrt{(k+1)/2}
  \tag{8}
\end{equation}
Для воздуха $k = 1.4$; $R = 29.27$; $g = 9.81$ м/с$^2$, поэтому
\addtocounter{equation}{1}
\begin{equation}\label{eq:129}
  a_{0} = \sqrt{kRT_0} = 20.1\sqrt{T_0};\ \ a_{\text{кр}} = a_{0}\sqrt{\frac{2}{k+1}} = 18.3\sqrt{T_0} 
  \tag{9}
\end{equation}
К этому можно добавить, что 
\addtocounter{equation}{1}
\begin{equation}\label{eq:1210}
  v_{\text{max}} = 2.23a_{0} = 44.8\sqrt{T_0}
  \tag{10}
\end{equation}
Это вытекает из
\[
  \frac{T_{0}-T}{T_{0}} = \frac{v^2}{kRT_0}\frac{k-1}{2} = \frac{k-1}{2}\frac{v^2}{a_{0}^{2}}
\]
при $T\rightarrow 0,\ v\rightarrow v_{\text{max}}$, т.е.
\[
  1 = \frac{v_{\text{max}}^{2}}{a_{0}^{2}}\frac{k-1}{2},\ \ v_{\text{max}} = a_{0}\sqrt{5} \approx 2.23a_0
\]
\\
2. Рассмотрим вопрос об обтекании клина сверхзвуковым потоком газа.
\begin{wrapfigure}[7]{l}{0.6\linewidth} 
  \includegraphics[width=100mm]{pics/1-2-1.png}
  \caption{}
  \label{1-2-1}
\end{wrapfigure}
Теоретическое рассмотрение и опыт показывают, что при натекании на клин с углом при вершине $2\delta$ сверхзвукового потока с числом Маха $M = v_{1}/a_{1} > 1$ к кромке клина присоединяется плоский косой скачок (ударная волна) с наклоном к направлению потока под некоторым углом $\varepsilon$ или под углом $\beta$ по отношению к поверхности клина. При этом поток газа, проходя через косой скачок, поворачивается на угол $\delta$ и идет параллельно поверхности клина [1].\\
Законы сохранения массы, энергии, а также теорема об изменении количества движения позволяют рассчитать все параметры об изменении количества движения потока за скачком, если известны параметры до скачка и угол клина. Обратимся к рис. \ref{1-2-2}.\\
\newpage
Пусть однородный сверхзвуковой поток со скоростью $v_1$ натекает на плоский косой скачок $l-l$, поворачивается на угол $\delta$ и течет за скачком со скоростью $v_2$.\\
\begin{wrapfigure}[9]{l}{0.65\linewidth} 
  \includegraphics[width=100mm]{pics/1-2-2.png}
  \caption{}
  \label{1-2-2}
\end{wrapfigure}
$v_{n_1},\ v_{t_1},\ v_{n_2},\ v_{t_2}$ -- составляющие скоростей $v_1$ и $v_2$ по нормали к скачку и вдоль него; $p_{1},\ \rho_{1},\ T_{1}$ -- параметры состояния газе до скачка; $p_{2},\ \rho_{2},\ T_{2}$ -- те же параметры состояния газа после скачка. За единицу времени материальная частица, занимавшая объем с боковым сечением $AOO'B$ займет с боковым сечением $OA'B'O'$. Так как перетекание через плоскость скачка может происходить только за счет нормальной составляющем скорости, то уравнение неразрывности в рассматриваемом случае имеет вид
\addtocounter{equation}{1}
\begin{equation}\label{eq:1211}
  \rho_{1}v_{n_1} = \rho_{2}v_{n_2}
  \tag{11}
\end{equation}
На основании теоремы об изменении количества движения для направлений вдоль скачка и перпендикулярно к нему имеем
\addtocounter{equation}{1}
\begin{equation}\label{eq:1212}
  \rho_{1}v_{n_1}v_{t_1} = \rho_{2}v_{n_2}v_{t_2},
  \tag{12}
\end{equation}
и
\addtocounter{equation}{1}
\begin{equation}\label{eq:1213}
  p_{1} + \rho_{1}v_{n_1}^{2} = p_{2} + \rho_{2}v_{n_2}^{2}
  \tag{13}
\end{equation}
Полная энергия частицы до скачка и после него одинакова, поэтому уравнение \eqref{eq:121dd} примет вид
\addtocounter{equation}{1}
\begin{equation}\label{eq:1214}
  \frac{k}{k-1}\frac{p_1}{\rho_1} + \frac{1}{2}v_{1}^{2} = \frac{k}{k-1}\frac{p_2}{\rho_2} + \frac{1}{2}v_{2}^{2}
  \tag{14}
\end{equation}
Если учесть, что $v/a = M,\ a = \sqrt{kp/\rho}$, получим следующие  соотношения между составляющими скоростей и углами, указанными на рис. \ref{1-2-1} и \ref{1-2-2}:
\addtocounter{equation}{1}
\begin{equation}\label{eq:1215}
  \begin{cases}
  v_{1}^{2} = v_{n_1}^{2} + v_{t_1}^{2} = k\frac{p_1}{\rho_1}M_{1}^{2}\\
  v_{2}^{2} = v_{n_2}^{2} + v_{t_2}^{2} = k\frac{p_2}{\rho_2}M_{2}^{2}
  \end{cases}
  \tag{15}
\end{equation}
\addtocounter{equation}{1}
\begin{equation}\label{eq:1216}
  \begin{aligned}
  v_{n_1} = v_{1}\sin\varepsilon,\ \ v_{t_1} = v_{1}\cos\varepsilon\\
  v_{n_2} = v_{2}\sin\beta,\ \ v_{t_2} = v_{2}\cos\beta
  \end{aligned}
  \tag{16}
\end{equation}
Из уравнения \eqref{eq:1212} с учетом равенства \eqref{eq:1211} имеем
\addtocounter{equation}{1}
\begin{equation}\label{eq:1217}
  v_{t_1} = v_{t_2}
  \tag{17}
\end{equation}
а из схемы на рис. \ref{1-2-2}
\addtocounter{equation}{1}
\begin{equation}\label{eq:1218}
  v_{1}/v_{2} = \cos\beta/\cos\varepsilon
  \tag{18}
\end{equation}
Путем ряда преобразование можно получить следующие выражения (см. [1]):
\addtocounter{equation}{1}
\begin{equation}\label{eq:1219}
  p_{2}/p_{1} = \frac{2k}{k+1}\left(M_{1}\sin^{2}\varepsilon - \frac{k-1}{2k}\right),
  \tag{19}
\end{equation}
\addtocounter{equation}{1}
\begin{equation}\label{eq:1220}
  \rho_{1}/\rho_{2} = \frac{2k}{k+1}\left(\frac{1}{M_{1}^{2}\sin^{2}\varepsilon} + \frac{k-1}{2}\right),
  \tag{20}
\end{equation}
\addtocounter{equation}{1}
\begin{equation}\label{eq:1221}
  \tang\varepsilon/\tang\beta = \frac{2}{k+1}\left(\frac{1}{M_{2}^{2}\sin^{2}\beta} + \frac{k-1}{2}\right),
  \tag{21}
\end{equation}
\addtocounter{equation}{1}
\begin{equation}\label{eq:1222}
  \frac{1}{\tang\delta} = \frac{k+1}{2}\left(\frac{M_{1}^{2}}{M_{1}^{2}\sin^{2}\varepsilon - 1} - 1\right)\tang\varepsilon,
  \tag{22}
\end{equation}
\addtocounter{equation}{1}
\begin{equation}\label{eq:1223}
  \varepsilon = \delta + \beta.
  \tag{23}
\end{equation}
Если параметры перед скачком $p_{1},\ \rho_{1},\ M_{1}$ и угол
клина $\delta$ (или $P_2$), то выражения \eqref{eq:1219}-\eqref{eq:1222} дают возможность определить параметры потока за скачком и угол наклона скачка.\\
II. \underline{Графическое решение задачи}\\
Учитывая, что $V_{t_1} = V_{t_2}$, можно картину скоростей перед и за
скачком стационарного потока газа, текущего вдоль оси $Ox$ и набегающего на клин с углом полураствора $\delta$, представить так, как указано на рис. \ref{1-2-3}.
\begin{figure}[!htp]
  \center{\includegraphics[width=140mm]{pics/1-2-3.png}}
  \caption{}
  \label{1-2-3}
\end{figure}
\\
Здесь по-прежнему индекс 1 соответствует параметрам потока до скачка, индекс 2 -- параметрам потока после скачка, $\delta$ -- угол поворота потока, $\varepsilon$ -- угол наклона скачка к направлению потока. Итак: $OA = V_{1}$, $OB\perp BA$ и $OB = V_{t_1} = V_{t_2}$, $BA = V_{n_1}$, точка С расположена $BA$, $OC = V_2$, $BC = V_{n_2}$, $u_{2},\ v_{2}$ -- проекции скорости $V_2$ на оси $Ox$ и $Oy$ соответственно.\\
Поставим себе целью найти уравнение, связывающее составляющие $u_{2},\ v_{2}$ с параметрами потока до скачка, например в виде $v_{2} = f(u_{2}, V_{1}, V_{\text{кр}})$.\\
Из уравнений \eqref{eq:1211} и \eqref{eq:1213} получаем
\[
  p_{2}-p_{1} = \rho_{1}V_{1}(V_{n_1}-V_{n_2}),
\]
или с учетом подобия треугольников $OBA$ и $DCA$
\addtocounter{equation}{1}
\begin{equation}\label{eq:1224}
  p_{2} = p_{1} + \rho_{1}V_{1}(V_{1}-u_{2}),
  \tag{24}
\end{equation}
Уравнение энергии \eqref{eq:1214} может быть записано в виде
\[
  \frac{k}{k-1}\frac{p_1}{\rho_1} + \frac{1}{2}V_{1}^{2} = \frac{k}{k-1}\frac{p_{0}}{\rho_0}
\]
он из \eqref{eq:129}
\[
  V_{\text{кр}}^{2} = a_{0}\frac{2}{k+1} = kRT_{0}\frac{2}{k+1} = \frac{2k}{k+1}\frac{p_0}{\rho_0},
\]
т.е.
\[
  p_{0}/\rho_{0} = \frac{k+1}{2k}V_{\text{кр}}^{2}
\]
поэтому
\addtocounter{equation}{1}
\begin{equation}\label{eq:1225}
  p_{1} = \rho_{1}\left(V_{\text{кр}}^{2}\frac{k+1}{2k} - V_{1}^{2}\frac{k-1}{2k}\right),
  \tag{25}
\end{equation}
аналогично
\addtocounter{equation}{1}
\begin{equation}\label{eq:1226}
  p_{2} = \rho_{2}\left(V_{\text{кр}}^{2}\frac{k+1}{2k} - V_{2}^{2}\frac{k-1}{2k}\right) = \rho_{2}\left(\frac{k+1}{2k}V_{\text{кр}}^{2} - \frac{k-1}{2k}(u^{2}+v^{2})\right),
  \tag{26}
\end{equation}
Кроме того, из указанного подобия треугольников и условия $\rho_{1}V_{n_1} = \rho_{2}V_{n_2}$ следует
\addtocounter{equation}{1}
\begin{equation}\label{eq:1227}
  \frac{\rho_1}{\rho_2} = \frac{V_{n_1}}{V_{n_2}} = \frac{V_{1}(V_{1} - u_{2})}{u_{2}(V_{1}-u_{2})-V_{2}^{2}}
  \tag{27}
\end{equation}
В самом деле, на основании подобия треугольников $OBA$ и $DCA$
\[
  V_{n_1}/V_{1} = (V_{1} - u_{2})/(V_{n_1} - V_{n_2})
\]
или
\[
\begin{aligned}
  \frac{V_{n_1}}{V_{n_2}} = \frac{V_{1}(V_{1}-u_{2})}{V_{n_1}V_{n_2} - V_{n_2}^{2}} = \frac{V_{1}(V_{1}-u_{2})}{V_{n_1}V_{n_2} - V_{2}^{2} + V_{1}^{2} - V_{n_1}^{2}} = \\
  = \frac{V_{1}(V_{1}-u_{2})}{V_{1}^{2} - V_{n_1}(V_{n_1}-V_{n_2}) - V_{2}^{2}} = \frac{V_{1}(V_{1}-u_{2})}{u_{2}(V_{1}-u_{2}) - V_{2}^{2}}
\end{aligned}
\]
Подставляя значения $p_{1},\ p_{2},\ \rho_{1},\ \rho_{2}$ из \eqref{eq:1226} - \eqref{eq:1227} в \eqref{eq:1224} получим уравнение в виде $f(u_{2}, v_{2}, V_{1}, V_{\text{кр}}) = 0$, дающее связь между составляющими скорости $u_{2}$ и $v_2$ за скачком и содержащее в виде параметра скорость потока $V_1$ перед скачком. Значения величин $V_{\text{кр}}$ и $k$, отражающих состав газа и его начальную температуру торможения, можно считать заданными. Разрешая это уравнение относительно $v_{2}^{2}$, получим
\addtocounter{equation}{1}
\begin{equation}\label{eq:1228}
  v_{2}^{2} = (V_{1}-u_{2})^{2} \frac{u_{2} - V_{\text{кр}}^{2}/V_{1}}{V_{\text{кр}}^{2}/V_{1} + \frac{2}{k+1}V_{1} - u_{2}}
  \tag{28}
\end{equation}
При определенных значениях $k$ и $V_{\text{кр}}$ для каждого значения скорости $V_1$ в плоскости годографа $(u, v)$ зависимость \eqref{eq:1228} представляет собой кривую второго порошка, определяющую геометрическое место точек концов вектора $V_2$. Эта симметричная относительно оси $Ou$ кривая называется строфоидой или угарной полярной. Откладываем на оси $Ou$ отрезки $OD = V_{\text{кр}}^{2}/V_{1}$, $OA = V_1$, $DE = 2V_{1}/(k+1)$. На отрезках $DA$ и $DE$ строим две окружности. Если из произвольной точки $F$ окружности $DFE$ провести прямую $FD$, то она пересечет малую окружность в точке, $H$. Опуская перпендикуляр $FC$ и проводя прямую $AH$, найдем точку  пересечения $B$, которая и будет концом вектора $V_2$, а $OC$ и $BC$ составляющими $u_2$ и $v_2$. Аналогично находятся точки 1, 2, 3, 4.\\
Докажем это. Из построения рис. \ref{1-2-4} вытекает $CF^{2} = DC\cdot CE$ и\\
$CB/CA = CD/CF$ или $CB^{2} = CA^{2}\cdot CD^{2} / CF^{2} = CA^{2}\cdot CD/CE$,\\
но $CA = V_{1}-u_{2}$,\ \ $CD=OC-OD=OC-V_{\text{кр}}/V_{1}$,\\
$CE=OD+DE-OC = V_{\text{кр}}^{2}/2 + 2V_{1}/(k+1)-OC$,\\
поэтому
\addtocounter{equation}{1}
\begin{equation}\label{eq:1228d}
  CB^{2} = (V_{1}-OC)^{2} \frac{OC - V_{\text{кр}}^{2}/V_{1}}{V_{\text{кр}}^{2}/V_{1} + 2V_{1}/(k+1) - OC}
  \tag{28$'$}
\end{equation}

\begin{figure}[!htp]
  \center{\includegraphics[width=170mm]{pics/1-2-4.png}}
  \caption{}
  \label{1-2-4}
\end{figure}

Если $CB = V_2$, а $OC = u_2$, то \eqref{eq:1228d} совпадает  \eqref{eq:1228}. При перемещении точки $F$ вдоль полуокружности $DFE$ точка $B$ опишет верхнюю половину ударной поляры 1-2-3-4 ... Ударная поляра пересекает ось $Ou$ в точках $D$ и $A$, причем точка $D$ соответствует минимальной скорости $V_{n_2}=V_{\text{кр}}^{2}/V_{1}$, соответствующей скорости потока, проходящего через прямой скачок уплотнения без изменения направления скорости. Точка $A$ является двойной, для нее $V_{2} = V_{1}$, т.е. скачок отсутствует. Участки кривой, лежащие между $u = V_1$ и асимптотой $u = V_{\text{кр}}^{2}/V_{1} + 2V_{1}/(k+1)$, соответствуют $V_{2}>V_{1}$, физического смысла они не имеют.\\
Для практических вычислений часто бывает удобнее пользоваться не скоростью $V$, а числом $M = V/a$ ($a$ -- скорость звука). В этом случае удобнее пользоваться для графического представления уравнением \eqref{eq:1228}, если все его члены разделить на $a_{1}^{2}$:
{\Large
\[
  \frac{v_{2}^{2}}{a_{1}^{2}} = \left(M_{1}-\frac{u_{2}}{a_{1}}\right)^{2}\frac{\frac{u_2}{a_1} - \frac{V_{\text{кр}}^{2}/a_{1}^{2}}{M_1}}{\frac{V_{\text{кр}}^{2}/a_{1}^{2}}{M_1} + \frac{2}{k+1}M_{1} - \frac{u_2}{a_1}}
\]
}

Вид рис. \ref{1-2-4} при этом не изменится, а соответствующие отрезки могут быть выражены через числа $M$: отрезок $OA$ представляет собой число $M_1$, отрезок $OB$ в том же масштабе представляет собой $M_{2}a_{2}/a_{1}$, а отрезки $GA$ и $GB$ выражают собой $M_{n_1}$ и $M_{n_2}a_{2}/a_{1}$. Здесь $a_1$ и $a_2$ -- скорости звука до скачка и за ним. Отношение этих скоростей можно определить по отрезкам $GA$ и $GB$, пользуясь уравнением \eqref{eq:1214}, которое при делении на $a_{1}^{2}$ принимает вид
\[
  \frac{2}{k-1} + M_{n_1}^{2} = \frac{2}{k-1}\cdot\frac{a_{2}^{2}}{a_{1}^{2}} + M_{n_2}\frac{a_{2}^{2}}{a_{1}^{2}}
\]
где учтено, что $V_{t_1} = V_{t_2}$, $a_{1}^{2} = k p_{1}/\rho_{1}$, $a_{2}^{2} = k p_{2}/\rho_{2}$. Заменяя $M_{n_1}$ и $M_{n_2}$ отрезками $GA$ и $GB$, получим
\[
  \frac{2}{k-1} + GA^{2} = \frac{2}{k-1}\cdot\frac{a_{2}^{2}}{a_{1}^{2}} + GB^{2}
\]
или
\[ 
  \frac{a_{2}^{2}}{a_{1}^{2}} = \frac{k-1}{2}(GA^{2} - GB^{2}) + 1.
\]
Отношение плотностей при переходе через скачок равно
\[
  \frac{\rho_1}{\rho_2} = \frac{V_{n_1}}{V_{n_2}} = \frac{GB}{GA}\cdot\frac{a_1}{a_2}
\]
а отношение давлений
\[
  \frac{p_2}{p_1} = \frac{\rho_2}{\rho_1}\cdot\frac{a_{2}^{2}}{a_{1}^{2}} = \left[ \frac{k-1}{2}(GA^{2}-GB^{2})+1\right] \frac{GA}{GB}\cdot\frac{a_1}{a_2}.
\]
\newpage
\noindent III. \underline{Описание экспериментальной установки}

Получение сверхзвуковой скорости осуществляется в аэродинамической трубе А-8, принципиальная схема которой изображена на рис. \ref{1-2-5}.

Сжатый воздух из газгольдеров практически без теплообмена
(вследствие кратковременности процесса), т.е. адиабатически 
расширяется и через систем трубопровода и задвижек поступает в 
фотокамеру, коробку сверхзвуковых сопловых вставок и рабочую часть.
В рабочей части происходит взаимодействие потока воздуха с 
обтекаемой моделью. Силовое воздействие потока на модель 
воспринимается весами (механическими или тензометрическими). Картина 
обтекания может наблюдаться с использованием различных методов 
визуализации (теневой метод, метод сажемаслянового покрытия, метод
нитей и т.д.). Для использования оптического метода с применением прибора ИАБ-451 в стенках рабочей части монтируются оптически однородные стекла. Спектры обтекания могут наблюдаться на матовом
стекле, фотографироваться или сниматься на киноленту.

\noindent IV. \underline{Порядок выполнения и оформления задачи}

1) Совместно с оператором подготовить аэродинамическую трубу к экспериментальной работе и провести эксперимент: поставить модель клина в рабочую часть аэродинамической трубы, закрыть все двери и люки контура трубы, подготовить прибор ИАБ-451, установить фотокамеру АФА или матовое стекло, открыть задвижку Лудло, при помощи быстродействующей задвижки выйти на заданный режим, снять отсчеты печатающим механизмом, сфотографировать спектр обтекания.

2) Пользуясь данными протокола печатающего механизма, величиной давления и температурой атмосферы и описанием задачи, построить ударную поляру и составить таблицу параметров сверхзвукового обтекания клина.\\
\begin{figure}[!htp]
  \center{\includegraphics[width=150mm]{pics/1-2-5.png}}
  \caption{}
  \label{1-2-5}
\end{figure}

Таблица заполняется в следующей последовательности:
\begin{enumerate}
  \item снимаются показания барометра $p_{\text{В}}$ и термометров в помещении $t_{\text{П}}$ и наружи $t_{\text{Н}}$, определяется атмосферное давление\footnote{Давление во всех случаях должно выражаться в одних и тех же единицах};
  \item $p_0$ определяется по показаниям измерителя полного давления;
  \item $T_0$\ \ $t_{\text{Н}}$ + 273 град."К";
  \item $\rho_0$ определяется по формуле \eqref{eq:122};
  \item $a_0$ определяется по формуле \eqref{eq:123} или \eqref{eq:129};
  \item $V_{\text{кр}}$ определяется по формуле \eqref{eq:129};
  \item $p_1$ определяется по показаниям измерителя статического давления;
  \item $M_1$ определяется по таблицам как функция отношения $p_{0}/p_{1}$;
  \item $\rho_{1},\ a_{1}$ определяются но таблицам при полученном значении $V_1$;
  \item $V_{1} = M_{1}a_{1}$
  \item $V_{2y},\ M_{2y}$ определяется по ударной поляре;
  \item $\varepsilon_{y}$ определяется по ударной поляре;
  \item $\varepsilon_{c}$ определяется по спектру обтекания;
  \item {\large$\Delta\varepsilon = \frac{\varepsilon_{c}-\varepsilon{y}}{\varepsilon_{c}}\cdot 100\%$};
  \item $\gamma$ определяется по спектру (угол между линией Маха за скачком и поверхностью клина);
  \item {\large $M_{2c} = 1/\sin\gamma$,\ \ $\Delta M_{2} = \frac{M_{2c}-M_{2y}}{M_{2c}}\cdot 100\%$.}\\
\end{enumerate}

\begin{center}
ТАБЛИЦА ДАННЫХ ДЛЯ ВЫПОЛНЕНИЯ ЗАДАЧИ:\\
$p_{\text{В}} = $\ \ \ \ \ \ \ \ мм рт.ст., \ \ 
$t_{\text{П}} = $\ \ \ \ \ \ \ \ град. C,\\
$p_{\text{атм}} = $\ \ \ \ \ \ \ \ кг/см$^2$, \ \ 
$t_{\text{Н}} = $\ \ \ \ \ \ \ \ град. C\\

\vspace{0.5cm}
\begin{tabular}{|l|p{4cm}|p{4cm}|p{4cm}|}
\hline
 & \makecell{$\delta = 5^o$} & \makecell{$\delta = 10^o$} & \makecell{$\delta = 15^o$}\\
\hline
 $p_0$, кг/см$^2$ & & &\\
\hline
 $T_0$, град "К" & & &\\
\hline
 $\rho$, кГс$^2\cdot$м$^{-4}$ & & &\\
\hline
 $a_0$, м/с & & &\\
\hline
 $V_{\text{кр}}$, м/с & & &\\
\hline
 $p_1$, кг/см$^2$ & & &\\
\hline
 $\rho_1$, кГс$^2\cdot$м$^{-4}$ & & &\\
\hline
 $a_1$, м/с & & &\\ 
\hline
 $M_1$ & & &\\
\hline
 $V_{1}$, м/с & & &\\ 
\hline
 $V_{2y}$, м/с & & &\\
\hline
 $M_{2y}$ & & &\\
\hline
 $\varepsilon_{y}$, град. & & &\\
\hline
 $\varepsilon_{c}$, град. & & &\\
\hline
 $\Delta\varepsilon$, \%\% & & &\\
\hline
 $\gamma$, град. & & &\\
\hline
 $M_{2c}$ & & &\\
\hline
 $\Delta M_{2}$, \%\% & & &\\
\hline
\end{tabular}
\end{center}
\newpage
\begin{center}
ЛИТЕРАТУРА
\end{center}
1. Черный Г.Г. Газовая динамика. М.: Наука, 1988. С.297-306.\\
2. Ферри А. Аэроданамика сверхзвуковых течений. М.-Л.: ГИТТЛ,
Госиздат технико-теоретической литературы, 1952. С.52-63.\\
3. Рахматулин Х.А., Сагомонян А.Я., Бунимович А.И., Зверев И.Н.
Газовая динамика. М.: Высшая школа, 1965. С.318-326.\\
4. Зауэр Р. Введение в газовую динамику. М.-Л.: ГИТТЛ, 1947.
C.115-120.

\newpage
\anonsubsection{Задача 3. Поперечные колебания бруса}
Целью работы является определение собственных частот и нормальных форм изгибных колебаний стержней, а также определение модуля продольной упругости методом колебаний.

\noindent I. \underline{Описание явления}\\
Нормальные формы и собственные частоты изгибных колебаний бруса

Механическая система с одной степенью свободы, выведенная из состояния равновесия и освобожденная затем от действия внешних сил, может совершать свободные гармонические колебания с определенной частотой, называемой собственной частотой. 

Примером могут служить колебания математического маятника, колебания груза, подвешенного на пружине малой массы и имеющего возможность перемещаться только в вертикальном направлении, крутильные колебания тонкого стержня с массой на конце и т.д.

В последних двух примерах пружина и стержень ввиду их малой массы играют роль только восстанавливающих усилий, упругих связей; возникающими в них инерционными усилиями можно пренебрегать по сравнению с силой инерции груза.

В теоретической механике показано, что механическая система с $n$ степенями свободы имеет $n$ собственных частот, соответствующих $n$ видам нормальных колебаний, так что перемещение каждой точки системы при свободных колебаниях представляется в виде геометрической суммы перемещений, которые получила бы точка при каждом из нормальных колебаний. При одном только нормальном колебании каждая точка системы совершает гармоническое колебание. Таким образом, \underline{нормальным колебанием} материальных точек называется такое свободное движение, при котором каждая точка совершает простое гармоническое колебание, причем частоты колебаний всех точек одинаковы и все точки колеблются в одной фазе.

Упругое тело, рассматриваемое как материальный континуум, имеет бесчисленное множество степеней свободы: каждая точка из бесчисленного множества материальных точек имеет три степени свободы, а взаимодействие ее с другими точками образует упругие связи. Поэтому упругое тело обладает бесчисленным множеством собственных частот, соответствующих бесчисленному множеству нормальных колебаний. Вектор перемещения точек упругого тела при каком-то $n$-ом нормальном колебании $U_n$ представляется, следовательно, таким образом:
{\Large
\[
  \overline{U_n} = \overline{u}(x, y, z) e^{i\omega_{n} t}
\]
}
где $\omega_n$ -- постоянная, определяющая собственную частоту. Из 
этого выражения для $\overline{U_n}$, видно, что форма тела при нормальном колебании изменяется во времени подобно самой себе. Произвольное свободное колебание можно представить в виде суммы нормальных колебаний:
{\Large
\[
  \overline{U}(x,y,z,t) = \sum\limits_{n}\overline{U_n}(x,y,z) e^{i\omega_{n} t}.
\]
}
Возможность представления сложных движений упругого тела при свободных колебаниях в виде результата наложения нормальных колебаний соответствует возможности разложения функции в тригонометрический ряд (теорема Фурье).

Если на упругое тело действует периодическая внешняя нагрузка с частотой, равной одной из собственных частот тела, то имеет место \underline{резонанс} -- амплитуда колебаний тела при отсутствии диссипативных сил будет возрастать теоретически до бесконечности, хотя практически всегда имеются диссипативные силы (внешние и
внутренние), препятствующие безграничному возрастанию амплитуды,
явление резонанса может привести к возникновению недопустимо
больших напряжений и к разрушению, особенно в случае резонанса на одной из низших собственных частот. В других случаях, наоборот, пользуются явлением резонанса для раскачки тел. Отсюда ясно, почему важно определение собственных частот и соответствующих им нормальных форм колебаний упругих тел.\\
\noindent II. \underline{Теоретическая часть}\\
Изгибные колебания бруса

Примем при рассмотрении поперечных колебаний балки за основу гипотезу плоских сечений. При этом, материальные частицы, которые в недеформированном состоянии находились в плоскости ортогонального сечения балки, переходят в результате деформации в плоскость, ортогональную серединному волокну, длина которого не меняется (это волокно называется нейтральным).
\begin{figure}[!htp]
  \center{\includegraphics[width=160mm]{pics/1-3-1.png}}
  \caption{}
  \label{1-3-1}
\end{figure}
На рис. \ref{1-3-1} $Q(x)$ -- перерезывающая сила; $M(x)$ -- изгибающий момент; $q(x)$ -- внешняя сила, приходящаяся на единицу длины.

Для равновесия балки необходимо ввести вместо мысленно отброшенной части балки (рис. \ref{1-3-1}, а) некоторую силу $Q$ (перерезывающая сила). Кроме силы $Q$ в сечении действует изгибающий момент $М$ (риc. \ref{1-3-1}, б), обусловленный растягиванием внешних волокон $A'B'$ и сжатием внутренних -- $A''B''$. В силу гипотезы плоских сечений распределение деформаций по толщине балки имеет вид
\addtocounter{equation}{1}
\begin{equation}\label{eq:131}
  \varepsilon(y_{0}) = y_{0}/R
  \tag{1}
\end{equation}
где $y_0$ -- лагранжева координата рассматриваемого волокна, значение $y_{0}=0$ соответствует нейтральному волокну; $R$ -- радиус кривизны нейтрального волокна.

Считая материал балки линейно-упругим $(\sigma = E\varepsilon)$, находим интегрированием по площади поперечного сечения результирующий момент
\addtocounter{equation}{1}
\begin{equation}\label{eq:132}
  M = EJ/R,
  \tag{2}
\end{equation}
где $E$ -- модуль Юнга материала балки; $J = \int\limits_{S}y_{0}^{2}ds$ -- момент инерции сечения балки относительно оси, ортогональной плоскости движения и проходящей через нейтральное волокно.

Величина $EJ$ называется жесткостью балки.

Для равновесия выделенного элемента балки длиной $dx$ должны быть выполнены условия равенства нулю главного вектора сил и моментов:
\addtocounter{equation}{1}
\begin{equation}\label{eq:133}
  \begin{aligned}
  & \delta Q = q(x)\delta x\\
  & Q\delta x = \delta M
  \end{aligned}
  \tag{3}
\end{equation}

Пусть $y(x)$ -- уравнение нейтрального волокна. Для малых прогибов $y'' \approx 1/R$. Тогда из \eqref{eq:133} и \eqref{eq:132} получим уравнение равновесия балки
\addtocounter{equation}{1}
\begin{equation}\label{eq:134}
  EJ\pdfrac{^{4}y}{x^4} = q(x)
  \tag{4}
\end{equation}

В случае свободного движения балки внешними силами являются силы инерции. В этом случае $y=y(x,t)$ т.е. прогиб является функцией времени, а $q(x,t) = -\rho S_{0} \partial^{2}y/\partial t^2$ (при свободных колебаниях согласно принципу Даламбера внешнюю распределенную нагрузку $q(x,t)$ следует заменить силами инерции). С учетом этого из \eqref{eq:134} получаем уравнение свободных колебаний балки
\addtocounter{equation}{1}
\begin{equation}\label{eq:135}
  EJ\pdfrac{^{4}y}{x^4} + \rho S_{0}\pdfrac{^{2}y}{t^2} = 0,
  \tag{5}
\end{equation}
где $S_0$ -- площадь сечения.

Если балка совершает нормальное изгибное колебание, то каждая точка его оси колеблется гармонически, причем частоты колебаний всех точек одинаковы и колебания совершаются в одной фазе, поэтому для обнаружения собственных частот и нормальных форм колебаний ищем частное решение уравнения \eqref{eq:135} в виде
\addtocounter{equation}{1}
\begin{equation}\label{eq:136}
  y(x, t) = e^{i\omega t} Y(x)
  \tag{6}
\end{equation}
где $Y(x)$ -- функция, характеризующая форму собственных колебаний (вид изогнутой оси), $\omega$ -- частота колебаний.

Подстановка \eqref{eq:136} в \eqref{eq:135} приводит к обыкновенному дифференциальному уравнению
\addtocounter{equation}{1}
\begin{equation}\label{eq:137}
  \frac{d^{4}Y(x)}{dx^4} - k^{4}Y(x) = 0,
  \tag{7}
\end{equation}
где
\addtocounter{equation}{1}
\begin{equation}\label{eq:13star}
  k^{4} = \omega^{2}\rho S_{0} / (EJ).
  \tag{*}
\end{equation}

Общим решением этого уравнения является функция
\addtocounter{equation}{1}
\begin{equation}\label{eq:138}
  Y(x) = C_{1}\text{ch}(kx) + C_{2}\text{sh}(kx) + C_{3}\cos (kx) + C_{4}\sin (kx)
  \tag{8}
\end{equation}

Если балка безгранична по длине и имеет опор, то ничего другого о нормальных колебаниях сказать нельзя: никаких ограничений на частоту колебаний $\omega$ не налагается, а нормальная форма колебаний ограничена только видом \eqref{eq:138} при неопределенных коэффициентах $C_{1},\ C_{2},\ C_{3},\ C_{4}$. Следовательно, для такого бруса (балки) любая частота может быть собственной, иначе говоря безграничный брус имеет непрерывный спектр собственных частот.

В брусе конечной длины имеются определенные граничные условия (условия закрепления), которые ограничивают произвол постоянных $C_{1}$, $C_{2}$, $C_{3}$, $C_{4}$; для этих величин получаются из граничных условий линейные однородные алгебраические уравнения, которые содержат в коэффициентах величину $k$, связанную с частотой $\omega$. Из условия наличия нетривиальных решений этой системы уравнений (иначе ни о каких колебаниях говорить нельзя) получается уравнение, вообще говоря трансцендентное, для $\omega$, называемое уравнением частот. Таким образом, наличие границ делает спектр собственных частот дискретным.

Так как из системы однородных уравнений, в коэффициенты которой подставляются корни уравнения частот, определяются только отношения искомых постоянных, например $C_{1} / C_{4}$, $C_{2} / C_{4}$, $C_{3} / C_{4}$, то форма нормального колебания определяется с точностью
до постоянного множителя. Это соответствует тому, что форма изогнутой оси при нормальном колебании изменяется во времени подобно самой себе, причем амплитуда колебаний остается неопределенной.

При разных способах закрепления концов получаются различные уравнения частот и различные формы нормальных колебаний.

Рассмотрим случай, когда балка свободна на одном конце и жестко заделана на другом.

Пусть конец балки $x = 0$ жестко заделан, а конец $x = L$ -- свободен. В заделке прогиб и наклон касательной к изогнутой оси балки равны нулю, т.е.
\addtocounter{equation}{1}
\begin{equation}\label{eq:139}
  y(0, t) = \partial y / \partial x |_{x = 0} = 0.
  \tag{9}
\end{equation}
На другом конце балки равны нулю изгибающий момент и перерезывающая сила, т.е.
\addtocounter{equation}{1}
\begin{equation}\label{eq:1310}
  \begin{aligned}
  Q(L, t) = EJ \partial^{3} y / \partial x^{3} |_{x=L} = 0,\\
  M(L, t) = EJ\partial^{2} y / \partial x^{2} |_{x=L} = 0
  \end{aligned}
  \tag{10}
\end{equation}
Если на свободный торец поместить груз массы $m$, то первое условие в \eqref{eq:1310} заменится уравнением движения груза массы $m$:
\[
  m\pdfrac{^{2}y(L,t)}{t^2} = -Q(L,t),
\]
а второе -- сохранится, если пренебречь инерцией вращения груза.

Граничные условия \eqref{eq:139} и \eqref{eq:1310} позволяют получить для определения постоянных $C_{1}$, $C_{2}$, $C_{3}$, $C_{4}$ систему однородных уравнений
\addtocounter{equation}{1}
\begin{equation}\label{eq:1311}
  \begin{cases}
  C_{1} + C_{3} = 0\\
  C_{2} + C_{4} = 0\\
  C_{1}\text{ch}(kL) + C_{2}\text{sh}(kL) - C_{3}\cos (kL) - C_{4}\sin (kL) = 0\\
  m\omega^{2}[C_{1}\text{ch}(kL) + C_{2}\text{sh}(kL) + C_{3}\cos (kL) + C_{4}\sin (kL)] =\\
  \ \ = k^{3}EJ[C_{1}\text{sh}(kL) + C_{2}\text{ch}(kL) + C_{3}\sin (kL) - C_{4}\cos (kL)]
  \end{cases}
  \tag{11}
\end{equation}
которая имеет нетривиальное решение только в случае равенства нулю ее определителя. Условие равенства нулю определителя и приводит к уравнению частот:
\addtocounter{equation}{1}
\begin{equation}\label{eq:1312}
  m\omega^{2}\left[\text{sh}(kL)\cos(kL) - \text{ch}(kL)\sin (kL)\right] - k^{3}EJ\left[1 + \text{ch}(kL)\cos (kL)\right] = 0.
  \tag{12}
\end{equation}
При отсутствии груза $(m=0)$ уравнение \eqref{eq:1312} имеет вид
\addtocounter{equation}{1}
\begin{equation}\label{eq:1313}
  \text{ch}(kL)\cos (kL) + 1 = 0.
  \tag{13}
\end{equation}
Каждому действительному корню этого уравнения $k_{n}L$ соответствует значение собственной частоты колебаний. Корнями уравнения \eqref{eq:1313} являются
\[
  k_{1}L = 1.875;\ \ k_{2}L = 4.694;\ ... \ ;\ k_{n}L = \frac{2n-1}{2}\pi;\ ...
\]

Принимая во внимание обозначение \eqref{eq:13star}, для собственных 
частот колебаний получаем
\addtocounter{equation}{1}
\begin{equation}\label{eq:1314}
  \omega_{1} = \lambda\left(\frac{1.875}{L}\right)^{2};\ \ \omega_{2} = \lambda\left(\frac{4.694}{L}\right)^{2};\ ...
  \tag{14}
\end{equation}
где $\lambda^{2} = EJ/(\rho S_{0})$.

Из системы уравнений \eqref{eq:1311} для каждого значения $k_{n}L$ можно определить отношение $C_{3}/C_{4}$:
{\large
\[
  \left(\frac{C_3}{C_4}\right)_{n} = -\frac{\text{sh}(k_{n}L) + \sin(k_{n}L)}{\text{ch}(k_{n}L) + \cos(k_{n}L)},
\]
}
так что форма нормальных колебаний определяется с точностью до
постоянного множителя $A_n$:
{\large
\[
  \begin{aligned}
  A_{n}Y_{n}(x) = \left[\text{sh}(k_{n}L) + \sin(k_{n}L)\right]\left[\text{ch}(k_{n}x) - \cos(k_{n}x)\right] - \\
   - \left[\text{ch}(k_{n}L) + \cos(k_{n}L)\right]\left[\text{sh}(k_{n}x) - \sin(k_{n}x)\right].
  \end{aligned}
\]
}

При каждом нормальном колебании имеется некоторое число точек оси, которые остаются неподвижными. Эти точки называются \underline{узлами} колебаний. Те точки между узлами, в которых амплитуда колебаний максимальна, называются \underline{пучностями}. Узел отличается от точки защемления бруса тем, что помещение в узле шарнирной опоры не приведет к возникновению опорной реакции. Поэтому, если положение узлов какой-либо конструкции известно, то удобно точки опоры располагать в узлах: инерционные усилия в этом случае не будут передаваться основанию.

Положение узлов определяется из уравнения $Y_{n}(x) = 0$, т.е. из уравнения
\addtocounter{equation}{1}
\begin{equation}\label{eq:1315}
  \begin{aligned}
  \left[\text{sh}(k_{n}L) + \sin(k_{n}L)\right]\left[\text{ch}(k_{n}x) - \cos(k_{n}x)\right] - &\\
   - \left[\text{ch}(k_{n}L) + \cos(k_{n}L)\right]\left[\text{sh}(k_{n}x) - \sin(k_{n}x)\right] = 0.&
  \end{aligned}
  \tag{15}
\end{equation}
Вид изогнутой оси (нормальная форма) для первых трех нормальных
колебаний показан схематически на рис. \ref{1-3-2}.

При колебаниях с собственной частотой $\omega_{1}$ неподвижной остается только точка закрепления (рис. \ref{1-3-1}, а). При колебаниях с частотой $\omega_2$ имеется один узел (рис. \ref{1-3-1}, б). При колебаниях с частотой $\omega_3$ появятся две узловых точки (рис. \ref{1-3-2}, в). Подсчеты показывают (см., например, [2]), что положения узлов определяются следующими числами, дающими отношение расстояния узла от свободного конца балки к длине балки.

\begin{center}
\begin{tabular}{p{7cm} p{2.5cm} p{2.5cm} p{2.5cm}}
  Порядковый номер\\ нормального колебания & \multicolumn{3}{c}{Положения узлов колебаний} \\
  & 1-й & 2-й & 3-й\\
  I & - & - & -\\
  II & 0.226 & - & -\\
  III & 0.132 & 0.499 & -\\
  IV & 0.094 & 0.356 & 0.6439\\
\end{tabular}
\end{center}
\begin{figure}[!htp]
  \center{\includegraphics[width=160mm]{pics/1-3-2.png}}
  \caption{}
  \label{1-3-2}
\end{figure}
\newpage
По наличию этих узловых точек можно в опытах судить о том, что брус совершает одно из нормальных колебаний.

\noindent III. \underline{Экспериментальная часть}

1. Описание экспериментальной установки.\\
Балка 1 длины $L$ (рис. \ref{1-3-3}) закреплена вертикально одним
концом в массивной подставке 2. На другом конце балки может быть закреплен груз 3 массой $m$. В некотором сечении балки наклеен постоянный магнит 4. При поперечных колебаниях балки магнит индуцирует ток в индуктивном датчике 5, расположенном на штативе 6. Электрический ток из цепи датчика, возникающий в результате индукции, подается на вход осциллографа. Основной измеряемой величиной является период собственных колебаний. Вблизи защемленного конца расположим электромагнит 7, после чего изменением частоты задающего генератора стандартных частот 8 добиваются совпадения одной из собственных частот колебаний стержня. О наступлении резонанса судят по появлению четко выраженных узлов колебаний и по максимальному возрастанию амплитуды. Форма нормальных колебаний (число узлов) указывает на порядковый номер собственной частоты.

Расстояния узлов от концов стержня измеряются линейкой.
\begin{figure}[!htp]
  \center{\includegraphics[width=170mm]{pics/1-3-3.png}}
  \caption{}
  \label{1-3-3}
\end{figure}

2. Определение модулей упругости по опытам на колебания\\
Для практических целей является существенно важными опыты по определению собственных частот и форм нормальных колебаний конструкций или их элементов, теоретические расчеты, для которых бывают сложными.

Поскольку теория изгибных колебаний проверена на многих опытах и является надежной, опыты по изгибным колебаниям стержней могут служить также для другой важной цели -- для определения механических характеристик материалов.

Механические характеристики (модули упругости, предел упругости и т.п.) зависят в той или иной мере от скорости деформаций. В опытах на колебания, регулируя частоту вынуждающей силы и амплитуду колебаний, можно в довольно широких пределах изменять скорость деформации.

В формулы для определения собственных частот колебаний 
входят модули упругости материала. В этих формулах собственные 
частоты можно считать известными, поскольку они определены из 
опытов, а искомыми величинами считать значения модулей упругости
при соответствующей скорости деформации. Например, в случае изгибных колебаний по формуле \eqref{eq:13star} находим
\[
  E = \frac{\omega^{2}\rho S_{0}L^{4}}{J(kL)^4}
\]
Если здесь вместо $\omega$ писать измеренную в опыте собственную
частоту $\omega_{n}$, а вместо $kL$ -- соответствующее значение корня уравнения частот, то для значения модуля продольной упругости при этой частоте получим
\addtocounter{equation}{1}
\begin{equation}\label{eq:1316}
  E_{n} = \frac{\omega_{n}^{2}\rho S_{0}L^{4}}{J(k_{n}L)^4}
  \tag{16}
\end{equation}
Для большинства металлов изменение модуля продольной упругости
очень слабо зависит от скорости деформации, так что из опытов
на колебания получают обычно значения $E$, близкие к тем, которые определяются в статических испытаниях.

3. Порядок выполнения работы
\begin{enumerate}
  \item[1)] Измерить продольный и поперечный размеры испытуемого
стержня.
  \item[2)] Включить генератор и осциллограф.
  \item[3)] Добиться совпадения вынужденной частоты с одной из частот собственных колебаний.
  \item[4)] Снять показания осциллографа.
  \item[5)] Определяются периоды и частоты собственных колебаний.
\end{enumerate}


\noindent III. \underline{Обработка и анализ результатов}

В протокол задачи включается:
\begin{enumerate}
  \item[1)] Экспериментальные значения собственных частот колебаний.
  \item[2)] Теоретические значения собственных частот колебаний, 
полученные по формулам \eqref{eq:1314}
  \item[3)] Построенные формы нормальных колебаний.
  \item[4)] Вычисленные по данным опытов значения модуля продольной 
упругости по формуле \eqref{eq:1316} $E_n$ для каждой из частот и средние значения, а также результаты сравнения с табличным значением.
\end{enumerate}
Результаты расчетов представить в таблицах:
\begin{center}
{\large
\begin{tabular}{|c|c|c|c|c|c|c|c|c|}
  \hline
  $N$ & $E$, Гпа & $\rho,\ \frac{\text{кг}}{\text{м}^{3}}$ & $L$, м & $R$, м & $J = \frac{\pi R^{4}}{4}$ & $\omega_{\text{э}} = \frac{2\pi}{T}$ & $\omega_{\text{теор}}$ & $E_{n}$\\
  \hline
  & & & & & & & & \\
  \hline
\end{tabular}
}
\end{center}
\vspace{0.5cm}
\begin{center}
ЛИТЕРАТУРА
\end{center}
1. Работнов Ю.Н. Механика деформируемого твердого тела. М.: Наука, 1979. С.207-213.\\
2. Рэлей. Теория звука. T.1.\\
3. Седов Л.И. Механика сплошной среды. Т.2. М.: Наука, 1973.
С.391-400.

\newpage
\anonsubsection{Задача 4. Распространение и отражение гидравлического прыжка}

Целью работы является ознакомление с экспериментальной и теоретической методикой исследования закономерностей распространения и отражения сильных разрывов в сплошной среде на примере наиболее наглядного и относительно просто описываемого локализованного переходного явления -- гидравлического прыжка в открытом водоеме.

\noindent I. \underline{Описание явления}

1. Для описания движения одной и той же материальной среды могут использоваться математические модели различного уровня детализации в зависимости от того, насколько существенно то или иное свойство данной материальной среды проявляется в конкретном ее движении. Существует множество практических задач, когда некоторое из свойств материальной среда, например вязкость, проявляется лишь в малой локализованной зоне, а всюду вне этой зоны движение является более простым и хорошо описывается при помощи упрощенной модели (не учитывающей отмеченное свойство). В таких случаях часто оказывается не обязательно (да и нецелесообразно) применять усложненную модель, пригодную для описания непрерывного движения всюду, в том числе и в указанных локальных зонах. Вместо этого можно ограничиться рассмотрением упрощенной модели, заменяя локальные переходные зоны более сложного взаимодействия \underline{поверхностями разрыва} решений системы уравнений, отвечающих упрощенной модели. Поверхности, на которых искомые функции непрерывны, но разрывны только некоторые их производные по координатам и времени, называются слабыми разрывами. Поверхности, при переходе через которые терпят разрыв сами искомые функции, называются поверхностями сильного разрыва. В качестве примера сильных разрывов в газообразных, жидких и твердых средах отметим такие, как скачки уплотнения перед быстролетящими телами, взрывные ударные и детонационные волны, гидравлические прыжки в открытом водоеме, сейсмические ударные волны в грунтах.

2. Течение тяжелой жидкости, например воды в открытых каналах, иногда сопровождается: возникновением скачкообразного повышения уровня свободной поверхности. Данное явление можно наблюдать в природе и в лабораторных условиях. Внешне это выглядит как достаточно крутая ступенька на поверхности воды, она называется гидравлическим прыжком или бором. В природе таким гидравлическим прыжком являются, например, океанические волны цунами, которые возникают в результате обширных, на протяжении сотен километров, подвижек дна океана, вызванных, например, землетрясением, или паводковые и разливные волны, образующиеся в результате разрушения дамбы или плотины.\\

\noindent II. \underline{Теоретическая часть}

1. Прямолинейный гидравлический прыжок, распространяющийся в покоящуюся воду на гладком горизонтальном дне
\begin{figure}[!htp]
  \center{\includegraphics[width=170mm]{pics/1-4-1.png}}
  \caption{Гидравлический прыжок}
  \label{1-4-1}
\end{figure}

Переходная зона -- "ступенька" уровня воды на рис. \ref{1-4-1} движется горизонтально слева направо с постоянной скоростью $D_1$ по покоящейся воде глубины $h_0$, вызывая общий подъем уровня воды на величину $\delta h = h_{1}-h_{0}$ и приводя весь слой жидкости в левом полупространстве в состояние поступательного горизонтального движения с постоянной скоростью $V_1$.

Внутри переходной зоны образуются сложные волновые структуры, происходят пространственные турбулентные движения и диссипативные процессы перемешивания жидкости, зачастую с участием пузырьков воздуха. Моделирование здесь осложнено необходимостью учета многих тонких свойств, присущих реальной жидкости. Однако всюду вне переходной зоны течение достаточно простое и справедлив гидростатический закон распределения давления по глубине:
\[
\begin{aligned}
&P_{0} = P_{a} + \rho g (h_{0}-z_{0})\ \ \text{-- справа от переходной зоны},\\
&P_{1} = P_{a} + \rho g (h_{1}-z_{0})\ \ \text{-- слева от переходной зоны}.\\
\end{aligned}
\]
Здесь $P_{a}$ -- атмосферное давление над свободной поверхностью; $\rho$ -- плотность воды; $g$ -- ускорение свободного падения в поле сил тяжести; $z_0$ -- вертикальная координата с началом на горизонтальном дне водоема.

При математическом описании гидравлического прыжка, как сильного разрыва, будем всюду в дальнейшем пренебрегать протяженностью переходной зоны и трением о дно водоема (рис. \ref{1-4-2}, а). Для получения условий на разрыве перейдем к подвижной инерциальной системе координат, связанной с фронтом гидравлического прыжка. Относительно новой (сопутствующей) системы координат обращенное движение жидкости будет установившимся и мы получим ситуацию, показанную на рис. \ref{1-4-2}, б. Здесь
\[
  x_{0} = x-tV_{+},\ \ z_{0}=z,\ \ D_{1} = \left(\frac{dx_0}{dt}\right)_{x=0} = V_{+}
\]
\newpage
\begin{figure}[!htp]
  \center{\includegraphics[width=170mm]{pics/1-4-2.png}}
  \caption{}
  \label{1-4-2}
\end{figure}

В системе координат $x,\ z$ на рис. \ref{1-4-2},а состояние жидкости по обе стороны разрыва характеризуется четырьмя параметрами: $h_{\pm}$, $V_{\pm}$. Однако они не могут быть произвольными, а должны удовлетворять соотношениям совместности на разрыве, которые вытекают из общих интегральных законов сохранения для жидких частиц, пересекающих поверхность разрыва $(V_{+}<0,\ V_{-}<0)$:



\anonsection{Раздел 2}
\anonsubsection{Задача 1. Волны разгрузки в гибких растяжимых нитях}
\anonsubsection{Задача 2. Изучение процесса разгона поршня в стволе пневматической установки}
\anonsubsection{Задача 3. Нелинейные волны сжатия (растяжения) сдвига в тонкостенной цилиндрической трубе}
\anonsubsection{Задача 4. Прямой экспериментальный метод построения ударных диаграмм сжатия грунтов}
\anonsubsection{Задача 5. Сверхзвуковое обтекание кругового конуса}
\anonsubsection{Задача 6. Соударение двух упругих тел}
\anonsubsection{Задача 7. Экспериментальное исследование подводного взрыва сферического заряда}
\anonsubsection{Задача 8. Влияние вращения цилиндра, находящегося в поперечном потоке, на распределение давления по его поверхности}


\end{document}
